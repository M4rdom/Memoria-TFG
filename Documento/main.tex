\documentclass[12pt, a4paper, twoside]{article}

%% Preamble
\usepackage{umatfgenglish}
\usepackage{blindtext}
\graphicspath{ {./images/} }



\begin{document}

\includepdf[noautoscale=true, width=\paperwidth]{Portada-Titulo-Contraportada/cover.pdf}

\newpage

\includepdf[noautoscale=true, width=\paperwidth]{Portada-Titulo-Contraportada/title.pdf}



%% Resumen
\begin{abstract}
    \blindtext[2]

	\bfseries{\large{Keywords:}}
\end{abstract}

\newpage

%% Abstract
\begin{abstract}
    \blindtext[2]

	\bfseries{\large{Palabras Clave:}}
\end{abstract}

\tableofcontents

%% Sections
\section{One \\}

\subsection{Motivation}
    \blindtext[2]
    
    \begin{figure}[ht]
      \centering
        \includegraphics[width=0.5\textwidth]{xp}
      \caption{A diagram showing the iterations of extreme programming.}
    \end{figure}
    
    \blindtext[4]

\section{Two \\}
    \blindmathpaper

%% Bibliography
\begin{thebibliography}{9}
    \bibitem{latexcompanion} 
    Michel Goossens, Frank Mittelbach, and Alexander Samarin. 
    \textit{The \LaTeX\ Companion}. 
    Addison-Wesley, Reading, Massachusetts, 1993.
    
    \bibitem{einstein} 
    Albert Einstein. 
    \textit{Zur Elektrodynamik bewegter K{\"o}rper}. (German) 
    [\textit{On the electrodynamics of moving bodies}]. 
    Annalen der Physik, 322(10):891–921, 1905.
\end{thebibliography}

\newpage

%% Apendices
\begin{umaappendices}
\section{Installation \\ Manual}
    
    \textbf{\large{Requirements:}}
    
    \blindtext

\end{umaappendices}









\newpage


%% Contraportada
\includepdf[noautoscale=true, width=\paperwidth]{Portada-Titulo-Contraportada/backcover.pdf}





\end{document}









