\documentclass[../main.tex]{subfiles}

\begin{document}
%% begin abstract format
\makeatletter
\renewenvironment{abstract}{%
    \if@twocolumn
      \section*{Abstract \\}%
    \else %% <- here I've removed \small
    \begin{flushright}
        {\filleft\Huge\bfseries\fontsize{48pt}{12}\selectfont Abstract\vspace{\z@}}%  %% <- here I've added the format
        \end{flushright}
      \quotation
    \fi}
    {\if@twocolumn\else\endquotation\fi}
\makeatother
%% end abstract format
\begin{abstract}

  This work aims to provide a new method for generating configuration files for building software containers. The goal of the project is to create a tool that allows developers to define a configuration and obtain a customized product, such as a Dockerfile. By using templates, we can apply best practices for building these files without requiring prior knowledge from the developer. To achieve this, we start by creating templates using Jinja syntax and the UVEngine variability resolution engine.

  The provided tool is a web application that enables developers to select a template and its version to resolve variability and generate a product. Currently, developers can define the configuration using the UVLS extension in VSCode and select the features of the service they wish to deploy.
  
  The project maintains a repository to store the templates, ensuring they can be updated or accommodate new ones. The system is divided into components to facilitate its maintenance. In conclusion, the effort made allows for the generation of configuration file templates with a high degree of customization compared to other existing methods.

\bfseries{\large{Keywords:}} A, B, C

\end{abstract}
\end{document}