\documentclass[../main.tex]{subfiles}

\begin{document}
%% begin abstract format
\makeatletter
\renewenvironment{abstract}{%
    \if@twocolumn
      \section*{Resumen \\}%
    \else %% <- here I've removed \small
    \begin{flushright}
        {\filleft\Huge\bfseries\fontsize{48pt}{12}\selectfont Resumen\vspace{\z@}}%  %% <- here I've added the format
        \end{flushright}
      \quotation
    \fi}
    {\if@twocolumn\else\endquotation\fi}
\makeatother
%% end abstract format
%% begin abstract format
\makeatletter
\renewenvironment{abstract}{%
    \if@twocolumn
      \section*{Resumen \\}%
    \else %% <- here I've removed \small
    \begin{flushright}
        {\filleft\Huge\bfseries\fontsize{48pt}{12}\selectfont Resumen\vspace{\z@}}%  %% <- here I've added the format
        \end{flushright}
      \quotation
    \fi}
    {\if@twocolumn\else\endquotation\fi}
\makeatother
%% end abstract format
\begin{abstract}

  Por revisar 
  %En la actualidad, el desarrollo de software y la implementación de servicios en la nube han experimentado un crecimiento exponencial. Las empresas buscan constantemente mejorar sus procesos de desarrollo y despliegue para ofrecer productos y servicios de alta calidad en el menor tiempo posible. En este contexto, herramientas como Docker y metodologías como la Integración Continua (CI) se han convertido en componentes esenciales para lograr estos objetivos.
%
  %Docker permite la creación de contenedores que encapsulan aplicaciones y sus dependencias, asegurando que el software se ejecute de manera consistente en cualquier entorno. Por otro lado, la Integración Continua facilita la detección temprana de errores mediante la automatización de pruebas y despliegues, lo que contribuye a un ciclo de desarrollo más ágil y eficiente.
  %
  %Este trabajo de fin de grado se centra en la generación automática de archivos de configuración para el despliegue de servicios en Docker. El objetivo principal es desarrollar una herramienta que simplifique y automatice el proceso de configuración, permitiendo a los desarrolladores centrarse en la creación de funcionalidades y mejoras en lugar de en tareas repetitivas y propensas a errores.
  %
  %A lo largo de este documento, se presentará una revisión de la literatura relacionada con Docker, Integración Continua y generación automática de configuraciones. Además, se describirá el diseño y la implementación de la herramienta propuesta, así como los resultados obtenidos y las conclusiones derivadas del proyecto.
  

\bfseries{\large{Palabras clave:}} SPL, Docker , Despliegues Servicios , Desarrollo web, Integración Continua , Configuración , feature models , Linea de Productos 

\end{abstract}
\end{document}


