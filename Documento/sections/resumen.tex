\documentclass[../main.tex]{subfiles}

\begin{document}
%% begin abstract format
\makeatletter
\renewenvironment{abstract}{%
    \if@twocolumn
      \section*{Resumen \\}%
    \else %% <- here I've removed \small
    \begin{flushright}
        {\filleft\Huge\bfseries\fontsize{48pt}{12}\selectfont Resumen\vspace{\z@}}%  %% <- here I've added the format
        \end{flushright}
      \quotation
    \fi}
    {\if@twocolumn\else\endquotation\fi}
\makeatother
%% end abstract format
%% begin abstract format
\makeatletter
\renewenvironment{abstract}{%
    \if@twocolumn
      \section*{Resumen \\}%
    \else %% <- here I've removed \small
    \begin{flushright}
        {\filleft\Huge\bfseries\fontsize{48pt}{12}\selectfont Resumen\vspace{\z@}}%  %% <- here I've added the format
        \end{flushright}
      \quotation
    \fi}
    {\if@twocolumn\else\endquotation\fi}
\makeatother
%% end abstract format
\begin{abstract}

  Este trabajo busca aportar un nuevo método para la generación de ficheros de configuración para la construcción de contenedores software.
  El fin del proyecto es generar una herramienta que permita a los desarrolladores definir una configuración y obtener un producto personalizado como puede ser un fichero \emph{Dockerfile}.
  Utilizando plantillas podemos aplicar sin conocimiento previo del desarrollador las prácticas recomendadas para la construcción de estos ficheros.
  Para lograr este objetivo se parte de la construcción de plantillas empleando la sintaxis de \emph{Jinja} y el motor de resolución de variabilidad \emph{UVEngine} para generar el producto.
  La herramienta proporcionada es una aplicación web que permite a los desarrolladores seleccionar una plantilla y versión de la misma para resolver la variabilidad y obtener un producto.
  Actualmente, los desarrolladores pueden definir la configuración haciendo uso de la extensión \emph{UVLS} en \emph{VSCode} y seleccionar las características del servicio que desean desplegar.
  El proyecto mantiene un repositorio para almacenar las plantillas con el objetivo de que estas puedan ser actualizadas o bien dar cabida a nuevas plantillas.
  El sistema se ha dividido en componentes para facilitar el mantenimiento del mismo. El esfuerzo realizado permite generar plantillas de ficheros de configuración con un alto grado de personalización
  en comparación con otros métodos existentes. 
  %Este trabajo busca aportar un nuevo metodo para la creacion de plantillas de ficheros de configuracion.  necesarios para la construccion de ficheros dockerfiles, con el fin de desplegar servicios en contenedores Docker. 
  %La herramienta pretende la generación automática de dichos archivos de configuración a partir de la creacion de configuracion validas , lo que permite a los desarrolladores definir las características de su servicio y obtener una configuración personalizada de forma rápida y sencilla.

  %Para lograr este objetivo, se ha utilizado una metodología de Línea de Productos de Software (SPL) para modelar las características de los servicios y generar los archivos de configuración correspondientes. La herramienta desarrollada permite a los desarrolladores definir las características de su servicio a través de un archivo de modelo y obtener una configuración personalizada de cada fichero como puede ser un dockerfile 
  %En la actualidad, el desarrollo de software y la implementación de servicios en la nube han experimentado un crecimiento exponencial. Las empresas buscan constantemente mejorar sus procesos de desarrollo y despliegue para ofrecer productos y servicios de alta calidad en el menor tiempo posible. En este contexto, herramientas como Docker y metodologías como la Integración Continua (CI) se han convertido en componentes esenciales para lograr estos objetivos.

  %Docker permite la creación de contenedores que encapsulan aplicaciones y sus dependencias, asegurando que el software se ejecute de manera consistente en cualquier entorno. Por otro lado, la Integración Continua facilita la detección temprana de errores mediante la automatización de pruebas y despliegues, lo que contribuye a un ciclo de desarrollo más ágil y eficiente.
  %
  %Este trabajo de fin de grado se centra en la generación automática de archivos de configuración para el despliegue de servicios en Docker. El objetivo principal es desarrollar una herramienta que simplifique y automatice el proceso de configuración, permitiendo a los desarrolladores centrarse en la creación de funcionalidades y mejoras en lugar de en tareas repetitivas y propensas a errores.
  %
  %A lo largo de este documento, se presentará una revisión de la literatura relacionada con Docker, Integración Continua y generación automática de configuraciones. Además, se describirá el diseño y la implementación de la herramienta propuesta, así como los resultados obtenidos y las conclusiones derivadas del proyecto.
  

\bfseries{\large{Palabras clave:}} Línea de Productos, Docker, Despliegues Servicios, Feature Models, Variabilidad

\end{abstract}
\end{document}

